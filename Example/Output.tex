% run the command ' lualatex -shell-escape Reference.tex ' twice in the terminal to visualize table of contents
\documentclass[twoside]{article}
\usepackage[utf8]{inputenc}
\usepackage[english]{babel}
\usepackage{geometry}
\usepackage{multicol}
\usepackage{minted}
\usepackage{python}
\usepackage[hidelinks]{hyperref}
\usepackage{fancyhdr}
\usepackage{listings}
\usepackage{pdfpages}
\usepackage{needspace}
\usepackage{sectsty}
\usepackage{array}
\usepackage{multirow}
\usepackage{longtable}
\usepackage{xcolor}
\usepackage{afterpage}
\usepackage{amssymb}
\usepackage{amsmath}
\usepackage[inline]{enumitem}
\usepackage{fontspec}

\geometry{letterpaper, portrait, left=0.5cm, right=0.5cm, top=1.8cm, bottom=1cm}

\sectionfont{\Huge\bfseries\sffamily}

\setminted{
    style=tango,
    breaklines=true,
    breakanywhere=true
}

\setlength{\headsep}{0.5cm}
\setlength{\columnsep}{0.5cm}
\setlength{\columnseprule}{0.01cm}
\renewcommand{\columnseprulecolor}{\color{gray}}

\pagestyle{fancy}
\pagenumbering{arabic}
\fancyhead{}
\fancyfoot{}
\fancyhead[LO,RE]{\textsf{First, solve the problem. Then, write the code.}}
\fancyhead[LE,RO]{\textsf{\leftmark}}
\fancyfoot[LE,RO]{\textbf{\textsf{\thepage}}}

\renewcommand{\headrulewidth}{0.01cm}
\renewcommand{\footrulewidth}{0.01cm}

\setlength{\parindent}{0em}
% column space
\setlength{\tabcolsep}{10pt} % Default value: 6pt
% upper and lower padding
\renewcommand{\arraystretch}{1.5} % Default value: 1

\setmainfont{Montserrat}

\newcommand{\fileTitleStyle}{\large\raggedright\underline}

\sectionfont{\centering\bfseries\Huge}
\subsectionfont{\centering\bfseries\LARGE}
\subsubsectionfont{\centering\bfseries\Large}
\paragraphfont{\centering\bfseries\large}

\begin{document}
\begin{multicols*}{2}
    \tableofcontents
\end{multicols*}
\begin{multicols*}{2}
\needspace{16\baselineskip}
\newpage
\sectionfont{\centering\bfseries\Huge}
\vspace{1em}
\section*{Graphs}
\markboth{GRAPHS}{}
\addcontentsline{toc}{section}{Graphs}
\vspace{3em}
\subsectionfont{\centering\bfseries\LARGE}
\vspace{0em}
\subsection*{Data Structures}
\addcontentsline{toc}{subsection}{Data Structures}
\vspace{2em}
\subsubsectionfont{\centering\bfseries\Large}
\subsubsectionfont{\fileTitleStyle}
\subsubsection*{Union Find}
\addcontentsline{toc}{subsubsection}{Union Find}
\begin{minted}{cpp}
struct UnionFind {
  int n;
  vector<int> dad, size;

  UnionFind(int N) : n(N), dad(N), size(N, 1) {
    while (N--) dad[N] = N;
  }
\end{minted}
\vspace{-12pt}
\needspace{4\baselineskip}
\begin{minted}{cpp}
  // O(lg*(N))
  int root(int u) {
    if (dad[u] == u) return u;
    return dad[u] = root(dad[u]);
  }
\end{minted}
\vspace{-12pt}
\needspace{8\baselineskip}
\begin{minted}{cpp}
  // O(1)
  void join(int u, int v) {
    int Ru = root(u), Rv = root(v);
    if (Ru == Rv) return;
    if (size[Ru] > size[Rv]) swap(Ru, Rv);
    --n, dad[Ru] = Rv;
    size[Rv] += size[Ru];
  }
\end{minted}
\vspace{-12pt}
\needspace{4\baselineskip}
\begin{minted}{cpp}
  // O(lg*(N))
  bool areConnected(int u, int v) {
    return root(u) == root(v);
  }
\end{minted}
\vspace{-12pt}
\needspace{4\baselineskip}
\begin{minted}{cpp}
  int getSize(int u) { return size[root(u)]; }

  int numberOfSets() { return n; }
};
\end{minted}

\needspace{9\baselineskip}
\subsectionfont{\centering\bfseries\LARGE}
\subsectionfont{\fileTitleStyle}
\subsection*{Articulation Points And Bridges}
\addcontentsline{toc}{subsection}{Articulation Points And Bridges}
\begin{minted}{cpp}
// APB = articulation points and bridges
// Ap = Articulation Point
// br = bridges, p = parent
// disc = discovery time
// low = lowTime, ch = children
// nup = number of edges from u to p
\end{minted}
\vspace{-12pt}
\needspace{5\baselineskip}
\begin{minted}{cpp}
typedef pair<int, int> Edge;
int Time;
vector<vector<int>> adj;
vector<int> disc, low, isAp;
vector<Edge> br;

void init(int N) { adj.assign(N, vector<int>()); }
\end{minted}
\vspace{-12pt}
\needspace{4\baselineskip}
\begin{minted}{cpp}
void addEdge(int u, int v) {
  adj[u].push_back(v);
  adj[v].push_back(u);
}
\end{minted}
\vspace{-12pt}
\needspace{15\baselineskip}
\begin{minted}{cpp}
int dfsAPB(int u, int p) {
  int ch = 0, nup = 0;
  low[u] = disc[u] = ++Time;
  for (int &v : adj[u]) {
    if (v == p && !nup++) continue;
    if (!disc[v]) {
      ch++, dfsAPB(v, u);
      if (disc[u] <= low[v]) isAp[u]++;
      if (disc[u] < low[v]) br.push_back({u, v});
      low[u] = min(low[u], low[v]);
    } else
      low[u] = min(low[u], disc[v]);
  }
  return ch;
}
\end{minted}
\vspace{-12pt}
\needspace{8\baselineskip}
\begin{minted}{cpp}
// O(N)
void APB() {
  br.clear();
  isAp = low = disc = vector<int>(adj.size());
  Time = 0;
  for (int u = 0; u < adj.size(); u++)
    if (!disc[u]) isAp[u] = dfsAPB(u, u) > 1;
}
\end{minted}

\needspace{6\baselineskip}
\subsectionfont{\centering\bfseries\LARGE}
\subsectionfont{\fileTitleStyle}
\subsection*{Topological Sort}
\addcontentsline{toc}{subsection}{Topological Sort}
\begin{minted}{cpp}
// vis = visited
vector<vector<int>> adj;
vector<int> vis, toposorted;
\end{minted}
\vspace{-12pt}
\needspace{4\baselineskip}
\begin{minted}{cpp}
void init(int N) {
  adj.assign(N, vector<int>());
  vis.assign(N, 0), toposorted.clear();
}

void addEdge(int u, int v) { adj[u].push_back(v); }
\end{minted}
\vspace{-12pt}
\needspace{12\baselineskip}
\begin{minted}{cpp}
// returns false if there is a cycle
// O(E)
bool toposort(int u) {
  vis[u] = 1;
  for (auto &v : adj[u])
    if (v != u && vis[v] != 2 &&
        (vis[v] || !toposort(v)))
      return false;
  vis[u] = 2;
  toposorted.push_back(u);
  return true;
}
\end{minted}
\vspace{-12pt}
\needspace{6\baselineskip}
\begin{minted}{cpp}
// O(V + E)
bool toposort() {
  for (int u = 0; u < adj.size(); u++)
    if (!vis[u] && !toposort(u)) return false;
  return true;
}
\end{minted}

\needspace{10\baselineskip}
\subsectionfont{\centering\bfseries\LARGE}
\vspace{0em}
\subsection*{Flow}
\addcontentsline{toc}{subsection}{Flow}
\vspace{2em}
\subsubsectionfont{\centering\bfseries\Large}
\subsubsectionfont{\fileTitleStyle}
\subsubsection*{Max Flow Min Cut Dinic}
\addcontentsline{toc}{subsubsection}{Max Flow Min Cut Dinic}
\begin{minted}{cpp}
// cap[a][b] = Capacity from a to b
// flow[a][b] = flow occupied from a to b
// level[a] = level in graph of node a
// iflow = initial flow, icap = initial capacity
// pathMinCap = capacity bottleneck for a path (s->t)
\end{minted}
\vspace{-12pt}
\needspace{5\baselineskip}
\begin{minted}{cpp}
typedef int T;
vector<int> level;
vector<vector<int>> adj;
vector<vector<T>> cap, flow;
T inf = 1 << 30;
\end{minted}
\vspace{-12pt}
\needspace{5\baselineskip}
\begin{minted}{cpp}
void init(int N) {
  adj.assign(N, vector<int>());
  cap.assign(N, vector<int>(N));
  flow.assign(N, vector<int>(N));
}
\end{minted}
\vspace{-12pt}
\needspace{7\baselineskip}
\begin{minted}{cpp}
void addEdge(int u, int v, T icap, T iflow = 0) {
  if (!cap[u][v])
    adj[u].push_back(v), adj[v].push_back(u);
  cap[u][v] += icap;
  // cap[v][u] = cap[u][v]; // if graph is undirected
  flow[u][v] += iflow, flow[v][u] -= iflow;
}
\end{minted}
\vspace{-12pt}
\needspace{17\baselineskip}
\begin{minted}{cpp}
bool levelGraph(int s, int t) {
  level.assign(adj.size(), 0);
  level[s] = 1;
  queue<int> q;
  q.push(s);
  while (!q.empty()) {
    int u = q.front();
    q.pop();
    for (int &v : adj[u]) {
      if (!level[v] && flow[u][v] < cap[u][v]) {
        q.push(v);
        level[v] = level[u] + 1;
      }
    }
  }
  return level[t];
}
\end{minted}
\vspace{-12pt}
\needspace{14\baselineskip}
\begin{minted}{cpp}
T blockingFlow(int u, int t, T pathMinCap) {
  if (u == t) return pathMinCap;
  for (int v : adj[u]) {
    T capLeft = cap[u][v] - flow[u][v];
    if (level[v] == (level[u] + 1) && capLeft > 0)
      if (T pathMaxFlow = blockingFlow(
          v, t, min(pathMinCap, capLeft))) {
        flow[u][v] += pathMaxFlow;
        flow[v][u] -= pathMaxFlow;
        return pathMaxFlow;
      }
  }
  return 0;
}
\end{minted}
\vspace{-12pt}
\needspace{9\baselineskip}
\begin{minted}{cpp}
// O(E * V^2)
T maxFlowMinCut(int s, int t) {
  if (s == t) return inf;
  T maxFlow = 0;
  while (levelGraph(s, t))
    while (T flow = blockingFlow(s, t, inf))
      maxFlow += flow;
  return maxFlow;
}
\end{minted}

\needspace{7\baselineskip}
\subsubsectionfont{\centering\bfseries\Large}
\subsubsectionfont{\fileTitleStyle}
\subsubsection*{Max Flow Min Cut Edmonds-Karp}
\addcontentsline{toc}{subsubsection}{Max Flow Min Cut Edmonds-Karp}
\begin{minted}{cpp}
// cap[a][b] = Capacity left from a to b
// iflow = initial flow, icap = initial capacity
// pathMinCap = capacity bottleneck for a path (s->t)
\end{minted}
\vspace{-12pt}
\needspace{4\baselineskip}
\begin{minted}{cpp}
typedef int T;
vector<int> level;
vector<vector<int>> adj, cap;
T inf = 1 << 30;
\end{minted}
\vspace{-12pt}
\needspace{4\baselineskip}
\begin{minted}{cpp}
void init(int N) {
  adj.assign(N, vector<int>());
  cap.assign(N, vector<int>(N));
}
\end{minted}
\vspace{-12pt}
\needspace{6\baselineskip}
\begin{minted}{cpp}
void addEdge(int u, int v, T icap, T iflow = 0) {
  if (!cap[u][v])
    adj[u].push_back(v), adj[v].push_back(u);
  cap[u][v] = icap - iflow;
  // cap[v][u] = cap[u][v]; // if graph is undirected
}
\end{minted}
\vspace{-12pt}
\needspace{19\baselineskip}
\begin{minted}{cpp}
// O(N)
T bfs(int s, int t, vector<int> &dad) {
  dad.assign(adj.size(), -1);
  queue<pair<int, T>> q;
  dad[s] = s, q.push(s);
  while (q.size()) {
    int u = q.front().first;
    T pathMinCap = q.front().second;
    q.pop();
    for (int v : adj[u])
      if (dad[v] == -1 && cap[u][v]) {
        dad[v] = u;
        T flow = min(pathMinCap, cap[u][v]);
        if (v == t) return flow;
        q.push({v, flow});
      }
  }
  return 0;
}
\end{minted}
\vspace{-12pt}
\needspace{14\baselineskip}
\begin{minted}{cpp}
// O(E^2 * V)
T maxFlowMinCut(int s, int t) {
  T maxFlow = 0;
  vector<int> dad;
  while (T flow = bfs(s, t, dad)) {
    maxFlow += flow;
    int u = t;
    while (u != s) {
      cap[dad[u]][u] -= flow, cap[u][dad[u]] += flow;
      u = dad[u];
    }
  }
  return maxFlow;
}
\end{minted}

\needspace{11\baselineskip}
\subsectionfont{\centering\bfseries\LARGE}
\vspace{0em}
\subsection*{Shortest Paths}
\addcontentsline{toc}{subsection}{Shortest Paths}
\vspace{2em}
\subsubsectionfont{\centering\bfseries\Large}
\subsubsectionfont{\fileTitleStyle}
\subsubsection*{Dijkstra}
\addcontentsline{toc}{subsubsection}{Dijkstra}
\begin{minted}{cpp}
// s = source
typedef int T;
typedef pair<T, int> DistNode;
T inf = 1 << 30;
vector<vector<int>> adj;
unordered_map<int, unordered_map<int, T>> weight;
\end{minted}
\vspace{-12pt}
\needspace{4\baselineskip}
\begin{minted}{cpp}
void init(int N) {
  adj.assign(N, vector<int>());
  weight.clear();
}
\end{minted}
\vspace{-12pt}
\needspace{7\baselineskip}
\begin{minted}{cpp}
void addEdge(int u, int v, T w, bool isDirected = 0) {
  adj[u].push_back(v);
  weight[u][v] = w;
  if (isDirected) return;
  adj[v].push_back(u);
  weight[v][u] = w;
}
\end{minted}
\vspace{-12pt}
\needspace{16\baselineskip}
\begin{minted}{cpp}
// ~ O(E * lg(V))
vector<T> dijkstra(int s) {
  vector<long long int> dist(adj.size(), inf);
  priority_queue<DistNode> q;
  q.push({0, s}), dist[s] = 0;
  while (q.size()) {
    DistNode top = q.top();
    q.pop();
    int u = top.second;
    if (dist[u] < -top.first) continue;
    for (int &v : adj[u]) {
      T d = dist[u] + weight[u][v];
      if (d < dist[v]) q.push({-(dist[v] = d), v});
    }
  }
  return dist;
}
\end{minted}

\end{multicols*}
\newpage
\sectionfont{\centering\bfseries\Huge}
\vspace{1em}
\section*{Extras}
\markboth{EXTRAS}{}
\addcontentsline{toc}{section}{Extras}
\vspace{3em}
\subsectionfont{\centering\bfseries\LARGE}
\vspace{0em}
\subsection*{Maths}
\addcontentsline{toc}{subsection}{Maths}
\vspace{2em}
\subsubsectionfont{\centering\bfseries\Large}
\subsubsectionfont{\fileTitleStyle}
\subsubsection*{Common Sums}
\addcontentsline{toc}{subsubsection}{Common Sums}
\vspace{1em}
\sffamily
\bgroup
\def\arraystretch{3}
{\normalsize
  \begin{center}
    \begin{tabular}{| >{\centering}m{5cm}| >{\centering}m{5cm}| >{\centering}m{5cm}|}
      \hline
      $\sum_{k = 0}^n k = \frac{n(n + 1)}{2}$
       &
      $\sum_{k = 0}^n k^2 = \frac{n(n + 1)(2n + 1)}{6}$
       &
      $\sum_{k = 0}^n k^3 = \frac{n^2(n + 1)^2}{4}$
      \tabularnewline \hline
      \multicolumn{2}{|c|}{$\sum_{k = 0}^n k^4 = \frac{n}{30} (n + 1)(2n + 1)(3n^2 + 3n - 1)$}
       &
      $\sum_{k = 0}^n a^k = \frac{1 - a^{n + 1}}{1 - a}$
      \tabularnewline \hline
      $\sum_{k = 0}^n ka^k = \frac{a[1 - (n + 1)a^n + na^{n + 1}]}{(1 - a)^2}$
       &
      \multicolumn{2}{c|}{
      $\sum_{k = 0}^n k^2a^k = \frac{a[(1 + a) - (n + 1)^2a^n + (2n^2 + 2n - 1)a^{n + 1} - n^2a^{n + 2}]}{(1 - a)^3}$
      }
      \tabularnewline \hline
      $\sum_{k = 0}^\infty a^k = \frac{1}{1 - a}, |a| < 1$
       &
      $\sum_{k = 0}^\infty ka^k = \frac{a}{(1 - a)^2}, |a| < 1$
       &
      $\sum_{k = 0}^\infty k^2a^k = \frac{a^2 + a}{(1 - a)^3}, |a| < 1$
      \tabularnewline \hline
      $\sum_{k = 0}^\infty \frac{1}{a^k} = \frac{a}{a - 1}, |a| > 1$
       &
      $\sum_{k = 0}^\infty \frac{k}{a^k} = \frac{a}{(a - 1)^2}, |a| > 1$
       &
      $\sum_{k = 0}^\infty \frac{k^2}{a^k} = \frac{a^2 + a}{(a - 1)^3}, |a| > 1$
      \tabularnewline \hline
      $\sum_{k = 0}^\infty \frac{a^k}{k!} = e^a$
       &
      $\sum_{k = 0}^n \binom{n}{k} = 2^n$
       &
      $\sum_{k = 0}^n \binom{k}{m} = \binom{n + 1}{m + 1}$
      \tabularnewline \hline
    \end{tabular}
  \end{center}
}
\egroup
\vspace{1em}
\begin{multicols*}{2}
\end{multicols*}
\subsubsectionfont{\centering\bfseries\Large}
\subsubsectionfont{\fileTitleStyle}
\subsubsection*{Logarithm Rules}
\addcontentsline{toc}{subsubsection}{Logarithm Rules}
\vspace{1em}
\sffamily
\bgroup
\def\arraystretch{2}
{\normalsize
\begin{center}
  \begin{tabular}{| >{\centering}m{5cm}| >{\centering}m{5cm}| >{\centering}m{5cm}|}
    \hline
    $\log_b(b^k) = k$
     &
    $\log_b(1) = 0$
     &
    $\log_b(X) = \frac{\log_c(X)}{\log_c(b)}$
    \tabularnewline \hline
    $\log_b(X \cdotp Y) = \log_b(X) + \log_b(Y)$
     &
    $\log_b(\frac{X}{Y}) = \log_b(X) - \log_b(Y)$
     &
    $\log_b(X^k) = k \cdotp \log_b(X)$
    \tabularnewline \hline
  \end{tabular}
\end{center}
}
\egroup
\vspace{1em}
\begin{multicols*}{2}
\end{multicols*}

\end{document}
